\section{Introduction}
\subsection{Motivation}
% - Importance of HPC
%   - With simulation and big data becoming a foundation of modern science
%   - Rise of ML, especially deep DL networks with over a trillion parameters
% - HPC at GWDG
%   - 4 HPC Systems
%     - SCC, Emmy, Grete, CARO
%   - SCC
%     - Oldest cluster
%     - User group: MPI and Uni
%     - Very heterogenous with X generations of CPUs and Y generations of GPUs
%       - Footnote which ones
%     - 2 locations (MDC, ???)
%     - Total compute power: ...
%   - Emmy
%     - Fourth Super Computer Generation of the HLRN
%     - 2 Providers: ZIB (Lise) and GWDG (Emmy)
%     - Top 500 Ranking: ...
%     - User Group: HLRN and NHR users
%     - Total Compute Power: ...
%   - Grete
%     - Top 500 Ranking: ...
%     - <https://gwdg.de/about-us/gwdg-news/2023/GN_3-2023_www.pdf>
%   - CARO
%     - Top 500 Ranking: ...
%     - User Group: DLR
%     - Im Betrieb seit 18.Juli 2022
%       - <https://www.dlr.de/de/aktuelles/nachrichten/2022/03/20220718_dlr-nimmt-neuen-supercomputer-in-goettingen-in-betrieb>
%     - Total compute power: ...
% - Importance of Monitoring
%   - Look at Slides
\subsection{Goals and Contributions}
% - Warum wir das eigentlich machen...
%   - bla bla Unifizierung des Monitorings
%   - Evaluation ob viable
% - Contributions:
%   - Evaluation der Skalierbarkeit von prometheus
%   - Evaluation der Skalierbarkeit von node exporter
%   - Performanceauswirkungen von node exporter auf laufende Programme
\subsubsection{Structure}
% Könnte man sogar bereits schreiben
\section{Background and Related Work}
\subsection{Monitoring}
% - typical architecture with graph
% - On Premise vs Cloud based
% - push vs pull
\subsubsection{Current Architecture: GWDG SCC}
% - TIG with Influxv2 and Flux
% - Still using InfluxQL
\subsubsection{Current Architecture: HLRN}
% - Grafana / Prometheus / Node Exporter 
\subsection{Time Series Databases}
\subsection{Prometheus}
% TODO prometheus is not just a tsdb, what is it
\subsection{TSDB Performance Evaluation}
% - TODO blabla there is a lot of nosql benchmarking research but not a lot on tsdb
% - TODO tsbs influx and victoriametrics
%   - Commercial incentives

\section{Benchmark Methodology}
% TODO bla bla which parts to benchmark
% Note that all code is available on GH
%   - Dont forget the second node exporter repo
\subsection{Setup}
% - HPC cluster and my laptop
% - Using default pre-built node exporter of version X
%   - Except for the patched metric throughput, where we forked at commit hash X
% - Using pre-built prometheus version 2.45.1 in an ubuntu 22.04 singularity container based on
%   the docker image available in the repo
% - 
\subsection{\texttt{node\_exporter}}
% - Welche Metriken wir beim starten aktivieren
\subsubsection{Metric Gathering}
% - Wie die Metriken mit green threads ge-fork-joined werden
% - TODO How we patched it instead of mocking for the most realistic version
%   - Show code in Appendix
\subsubsection{End to End}
% - We use two tools
%   - wrk TODO
%   - go-wrk TODO
% - All benchmarks are measured using `vmstat 1`
% - We have three types of benchmarks
%   - wrk seq
%     - Most realistic load, as one usually has a single prometheus
%     - simplest benchmark: we just blast as fast as possible with 1 thread
%   - wrk par
%     - It keeps the HTTP connections open for maximizing throughput.
%     - While it maximizes throughput, it is not realistic as prom wouldnt do that!
%     - scale along threads and HTTP connections
%   - go-wrk par
%     - it uses the same go `net` lib as prom
%     - It doesnt keep connections open but is highly parallel with go's green threading model
%     - Scale along go-threads
% - For the parallel benchmarks, we start it with some kind of parallelism for some amount
%   then increase parallelism in a step and so on and so fourth:
%   - We also scale along the axis of number of processe consumed by node exporter through the
%     GOMAXPROCS
%     - Note that even with n=1 go threads can still highly improve performance by fixing I/O idle
% - Fully automatic pipeline
\subsubsection{Jitter}
% - Why we think this could be a problem
%   - As mentioned before it spawns up several green threads
%   - Thus more memory footprint, thus potentially wasting the whole CPU cache
\subsection{Prometheus}
% - All benchmarks are measured using `vmstat 1`

% - We mocked node exporter
%   - TODO fastapi 0.104 uvicorn 0.24 single route
%   - TODO only integers, configurable amount of metrics, all nodes have same metrics
%   - TODO how we did randomness with urandom times mersenne warmup
%   - TODO how we count the requests to see whether sth is skipped
%     - Locking global counter
%     - asynccontextmanager
%     - unique filename through pid

% - High level (automatized) workflow.
%   - Give number of mock clients as cli param by providing a port range
%   - Create a dynamic prom cfg for those ports and a 10sec interval to bind-mount in via
%     singularity
%   - spawn all python mock processes in the port range, 1 process per port
%   - Start `vmstat 1`
%   - Start the singularity container with prometheus and the tmp config as well as the 
%     data directory binded in
%     - Explain that the data directory is needed for write ops in footnote
%   - sleep for the amount of the benchmark
%   - terminate prom, then vmstat, then all mock exporter
%     - Terminate == SIGTERM

\section{Results}
\subsection{\texttt{node\_exporter}}
\subsection{Prometheus}

\section{Discussion}
% Node Exporter:
% - we used /dev/urandom, not perlin noise. This is less realistic but less overhead
%   - Even more, the aforementioned tsbs has some predefined data gen
%     - Also they split up the step to avoid the overhead; for us too much IO overhead
%       - Note that we have a convert script to JSON in the repo

\section{Conclusion}
