%%%%%%%%%%%%%%%%%%%%%%%%%%%%%%%%%%%%%%%%%
% Presentation Template
% LaTeX Template
% Version 1.0 (2023-02-08)
%
% This template was adapted by:
% Jonathan Decker (jonathan.decker@uni-goettingen.de)
% From a template made by:
% Julian Kunkel (julian.kunkel@gwdg.de)
%
%%%%%%%%%%%%%%%%%%%%%%%%%%%%%%%%%%%%%%%%%
\documentclass[compress,aspectratio=169]{beamer}

% make sure the theme file is on this path
\usepackage{../assets/beamerthemeGoettingen}

\addbibresource{ref.bib}
\graphicspath{{../}{../assets/}}

% --- document configuration ---
\newcommand{\mytitle}{Evaluation of Time-Series Databases}
% Leave empty for no subtitle
\newcommand{\mysubtitle}{From InfluxDB to Elasticsearch}
\newcommand{\myauthor}{Lars Quentin}
\newcommand{\myauthorurl}{https://lquenti.de/}
\newcommand{\myvenue}{HPCSA}
% For example, use \today
\newcommand{\mydate}{\today}
% For example, Institute for Computer Science / GWDG
\newcommand{\myinstitute}{University of G\"ottingen}

\configuretitlepage


\begin{document}

\begin{frame}[plain]
	\titlepage
\end{frame}

\begin{frame}[t]{Table of contents}
  \tableofcontents[subsectionstyle=hide/hide]
\end{frame}

% --- slides begin ---

\section{Section 1}

\begin{frame}{Itemize}
  \begin{itemize}
    \item
  \end{itemize}
\end{frame}

\begin{frame}{Itemize within itemize}
    \begin{itemize}
        \item List
            \begin{itemize}
                \item Sub-bullet
            \end{itemize}
    \end{itemize}
\end{frame}

\begin{frame}{Block and negative space}
    \vspace*{-3em} % Make more space by moving everything up
    \begin{block}{Block}
        \begin{itemize}
            \item
        \end{itemize}
    \end{block}
\end{frame}


\section{Section 2}
\sectionIntroHidden % Show an outline of the current section with hidden subsections
%\sectionIntro % Show an outline of the current section with subsections

\begin{frame}{Includegraphics}
    \centering
    \includegraphics[width=0.6\textwidth]{assets/hps-logo.pdf}\\
    \source{Image source: \url{https://hps.vi4io.org/start}}
\end{frame}

\subsection{Subsec 1} % Subsections are used for highlighting parts of the top navigation bar, the names only show up in the sectionIntro
\begin{frame}{Pause}
    \begin{itemize}
        \item 1
        \pause  % Create separate slides in the pdf to show points one at a time
        \item 2
        \pause
        \item 3
    \end{itemize}
\end{frame}

\begin{frame}{Columns}
    \begin{columns}
    \begin{column}{0.5\textwidth}
    Left column
    \end{column}
    \begin{column}{0.5\textwidth}
    Right column % Can also be used for inserting images next to text
    \end{column}
    \end{columns}
\end{frame}

\begin{frame}{Tikz graphics}
    % Use tikz for placing images based on page coordinates
    \begin{tikzpicture}[remember picture,overlay]
      \node [xshift=-2.8cm,yshift=-1.8cm] at (current page.north east)
        {\includegraphics[width=4cm]{assets/hps-logo.pdf}};
    \end{tikzpicture}
    \source{Image source: \url{https://hps.vi4io.org/start}}
\end{frame}

\begin{frame}[fragile]{Code listings}
    % Use minted package for handling code listings on slides
    \texttt{Hello World} implemented in C++
    \begin{tcolorbox}[title=C++]
        \footnotesize\inputminted[xleftmargin=1em,linenos]{c++}{../assets/hello-world.cpp}
    \end{tcolorbox}

\end{frame}

\begin{frame}{Todonotes}
    \todo{Set todo reminders for yourself}
\end{frame}

\begin{frame}{Quotes}
    % Make a quote the central element of a slide to emphasize its importance
    \vspace*{\fill}
    \begin{quote}
        \centering\Large
        \enquote{Non-reproducible single occurrences are of no significance to science.}
    \end{quote}
    \vspace{1cm}
    \hspace*\fill{\small--- Karl Popper, The Logic of Scientific Discovery, 2002, p.\,66}
    \vspace*{\fill}

    \source{\cite{popperLogicScientificDiscovery2002}}
\end{frame}


\section{Section 3}
\begin{frame}{Last Frame}
\label{pg:lastpage} % Label on last frame to get the page number for footer

\end{frame}

\begin{frame}{References}
    % References slide in appendix
    \renewcommand*{\bibfont}{\normalfont\scriptsize}
    \printbibliography[heading=none]
\end{frame}

\end{document}
